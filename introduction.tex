\chapter{Introduction}

\Capinsert[4]{\textbf{B}}{ovinely} invasive brag; cerulean forebearance.
Washable an acre. To canned, silence in foreign.
Be a popularly. A as midnight transcript alike.
To by recollection bleeding. That calf are infant. In clause.
Buckaroo loquaciousness?  Aristotelian!
Masterpiece as devoted. My primal the narcotic. For cine?
In the glitter. For so talented. Which is confines cocoa accomplished.
Or obstructive, or purposeful.
And exposition? Of go. No upstairs do fingering.


The future of nuclear medicine would appear to be the paradigm of personalized medicine --- targeted radionuclide therapy to spare healthy tissue \cite{Mulford2005,Qaim201731}, and theranostic medicine, which pairs an imaging isotope with a therapeutic isotope (frequently, of the same element), to provide simultaneous, real-time dose delivery and verification, leading to drastic reductions in prescribed patient dose \cite{Muller2014,Bentzen2005,Srivastava2012}. 
Other variants of theranostic medicine exist, including pre-imaging for treatment planning, or delivery of a single compound with different radioelements for imaging/therapy where the inter-element biodistribution has been validated.  




% An accurate integrated proton current is one of the most important factors in performing high-fidelity cross section measurements.
% At the time of this work, the nondestructive beam current monitors in the LANSCE-IPF beamline had a  resolution of 100 nAh.
% For a low-current irradiation such as this work, where a nominal fluence of 200 nAh is desired, additional fluence sensitivity is thus needed to accurately normalize quantified EoB activities into cross sections.
% To this end, 

\vspace{1cm}

\section{Faceplate Marginalia}

The overarching goal of this project is to develop capabilities for bench-to-preclinical production of the novel emerging medical radionuclides 211At, 77Br, and 76Br at the University of Oslo (UiO), which are desired for personalized cancer therapeutic applications and PET/SPECT diagnostic imaging. These chemically complementary medical radionuclides are further desired for their ability to be generated in theranostic pairs, or even triplets, for next-generation combined therapeutic and imaging treatment applications. The lack of widespread access to clinically relevant quantities of these radionuclides has typically the greatest impediment to advancing their widespread use in pre-clinical studies. Building the capability for production of activity in such quantities will thus help to enable more development of these radionuclides for clinical application. Recent studies suggest that on average, nearly one in three individuals will be diagnosed with cancer during their lifetime. Current treatment options, including surgery, conventional cytotoxins, chemotherapy, and external beam radiation therapy, face several obstacles in effectively treating these diseases. Long-term survival is especially challenging for aggressive and invasive strains, as well as metastatic and recurrent cancers. In these cases, the cure may be worse than the disease itself, as the aggressive treatment approaches used to combat the spread of disease often cause significant side effects through widespread damage to organs and healthy tissues. 

It is clear that this is a fundamental, systemic problem for society, with inherently interdisciplinary approaches required for the development of next-generation solutions for treatment and detection.  One such emerging approach is that of targeted radionuclide therapy, which utilizes the intravenous delivery of a therapeutic radionuclide coupled with a “targeting vector” biomolecule, to precisely deliver a radioactive “payload” to the site of disease. Radionuclide therapy offers the potential benefits of both external beam radiotherapy (destruction of cancer cells by radiation-induced DNA damage) and conventional chemotherapy (systemic treatment throughout the body), without the associated side effects both of these methods commonly produce through accidental damage of healthy tissue.  In the process of radioactive decay, radionuclides deposit their payload of radiation/energy isotropically. This allows radionuclides to deliver a therapeutic dose in an approximately spherical volume around the site of each single nuclide, allowing them to kill a small number of surrounding cells, in addition to the directly targeted cell. The choice of a particular radionuclide gives the medical team control over the selectivity of this dose range, leading to the potential to “paint” a tumor with a “brush” of tuneable width. Similarly, candidates for targeting vehicles are chosen to systemically seek out cancerous cells throughout the body, thereby selectively delivering a dose only to the site of disease, sparing healthy tissue and organs throughout the body. More importantly, this allows the radionuclides to treat not only any primary tumor sites, but any other undetected metastases which may have spread throughout the body. Additionally, instead of a therapeutic radionuclide, one which emits either positrons or a single gamma-ray may be attached to the targeting vector, to detect the presence of cancerous cells through conventional PET or SPECT diagnostic imaging modalities. Vitally, this combination of radionuclides and targeting vectors is inherently modular in nature – for a given radionuclide, different vectors may be coupled to it, based on where the radionuclide is desired to be selectively delivered. Conversely, once a targeting vector is established, different radionuclide payloads can be attached to it, based on the range of dose desired for delivery, or for imaging instead.

The promise of these methodologies seeks to shift the paradigm of modern cancer diagnosis and treatment, especially when used in combination. The future of nuclear medicine would appear to be that of personalized medicine — targeted radionuclide therapy to spare healthy tissue, and theranostic medicine, which pairs a mixture of an imaging isotope with a therapeutic isotope to provide simultaneous, real-time dose delivery and verification, leading to drastic reductions in received patient dose. Relatively few radionuclides possess physical decay characteristics which make them desirable for this application, so exploratory research is heavily focused on a small number of emerging candidates. Candidate isotopes to meet these needs have been identified based on their chemical and radioactive decay properties1. In my PhD studies, I have lead a series of campaigns to perform targeted, high-priority measurements of thin-target cross sections and thick-target integral yields for many of these emerging novel medical radionuclides2,3. These efforts have been motivated by the need to improve existing nuclear data for these valuable production reactions, as well as to ultimately develop capabilities to produce several valuable radionuclides in sufficient quantities to facilitate the production of clinically relevant quantities of radioactivity. While this work has contributed to the development of new methods for precision measurements of the production of emerging medical radioisotopes, this work has primarily focused on those radionuclides with diagnostic applications. However, many of these same methods which I have helped develop may also be applied to begin investigations for the production of emerging therapeutic radionuclides as well. 

In selecting therapeutic radionuclides, a vital figure of merit is the linear energy transfer (LET, typically reported in keV/μm) of their decay radiation, which measures the energy deposition per unit length. Radionuclides with high LET radiation produce a high density of ionization events along their trajectories, which cause damage to the integrity of cells and their DNA. In addition, LET is inversely proportional to the radius over which this energy is deposited. Thus, high-LET radionuclides are prized for therapeutic potential, as their decay radiation produces high cellular lethality over a narrow region, leading to precise delivery of high dose, with minimal dose to surrounding cells. Historically, most conventional radionuclide therapy has been reliant upon radioisotopes which decay through β- particle emission, chiefly the radionuclides 89Sr and 131I. β- particles possess low LET (> 0.3 keV/μm) and long range (100–10,000 μm) compared to the 10–30 μm size of most human cells. As a result, β- particle therapy has had limited success outside of the treatment of large, solid tumor masses such as prostate cancers. This long range makes it difficult to deliver the to deliver the radiation doses needed for irreparable cellular damage to the disease without using high radionuclide concentrations, and in the process, often delivers a high dose to surrounding healthy tissue, as well as the rest of the body. 

The future of radionuclide therapy appears to be focused on the development of higher-LET isotopes. These fall into two major groups: those which decay by emission of an alpha particle (“alpha emitters”), as well as those which emit a cascade of Auger electrons in their decay (“Auger emitters”).  Many alpha emitters belong to the actinide series and other heavy elements, and possess long decay chains. This radiochemical behaviour has makes handling of many therapeutic alpha emitters challenging, which historically has been an impediment to production and preclinical development studies. However, their decay properties give them high therapeutic efficiency — the 5–10 MeV alpha particles commonly emitted possess an LET of 80–100 keV/μm and a range of 40–100 μm. For decay chains involving multiple alpha particle emissions, subsequent decays tend to be extremely short-lived, localizing the dose of the several alpha particles emitted. This makes alpha particles extremely lethal to human tissue and their short range (on the order of a single cell) significantly reduces the dose delivered to surrounding healthy tissue. Most Auger emitters are commonly radionuclides which undergo electron capture decay, leading to the emission of a cascade of low-energy (10 eV – 10 keV) Auger electrons and Coster–Kronig electrons. Such electrons possess an LET of 5–25 keV/μm, which corresponds to a range of 2–500 nm. In addition, due to the electron vacancy cascade mechanism behind their emission, most Auger emitters release between 5–20 electrons in the span of a few femtoseconds following the decay of a single radionuclide, leading to a massive accumulated dose in a volume comparable to the nucleus of a single cell. The magnitude of this energy transfer makes Auger emitters suitable for directly inducing double-strand breaks in the DNA of a targeted cancerous cell, from which it is nearly impossible for the cell to repair itself in a way which permits it to survive and divide.  While the range of Auger electrons prevents them from depositing dose into more than a few nm3 (the few immediately surrounding healthy cells), this extreme dose localization has created challenges in matching radionuclides with suitable targeting vectors.

Candidate isotopes to meet the needs for both targeted alpha and Auger therapies have been identified based on their chemical and radioactive decay properties. One of the most promising alpha emitters currently being developed is that of the radiohalogen 211At (t1/2=7.21 hr). Every 211At decay leads to the emission of either a 5.87 MeV alpha particle (through the 41.8% α decay branch of 211At), or a 7.45 MeV alpha particle (through every short-lived t1/2=0.516 s α decay in 211Po, fed by the 58.2% electron capture branch of 211At), as seen in Figure 1. These alpha particles have an average LET of 97 keV/μm, and range of 50–70 μm in tissue. In decay chains with multiple alpha emissions, the products along the decay chain tend to diffuse further and further away from the site of the original localized parent isotope, due to degradation of the targeting vector by the emitted alpha particles. However, for the simple decay scheme of 211At, little diffusion of the 211Po is observed. This is because 211Po decays rapidly enough following electron capture decay of 211At that the 211Po decays before it has a chance to diffuse far away from the site of the targeted disease. These favourable decay properties, along with the cellular lethality of both alpha particles, has made 211At attractive for a number of therapeutic applications, including the treatment of ovarian and gynaecological cancers, as well as glioblastoma and other recurrent brain cancers. 211At may also be easily coupled to monoclonal antibodies (mAb) as a targeting vector for selective delivery of the radionuclide.

Another pair of promising radiohalogens are the bromine isotopes 76Br and 77Br. 77Br (t1/2 = 57.04 hr) is an emerging Auger therapy agent, which produces a cascade of approximately, on average, seven low-energy electrons with an average LET of 14 keV/μm: 9.67 and 1.32 keV Auger electrons (with ranges of approximately 3.1 and 0.9 μm in tissue, respectively), and several 20–80 eV Coster–Kronig electrons, with ranges of 1–7 μm in tissue. Like any Auger emitter, the radionuclide must be localized within a cancerous cell for maximum cytotoxicity, so it should be coupled to a biomolecule capable of penetrating the cellular membrane.  Its lifetime permits uptake of targeted 77Br by either labelling of various targeted proteins, or direct incorporation into cellular DNA by radiobromine-labelled nucleosides. The specificity and high lethality of its electron cascade has made 77Br an attractive candidate for the treatment of a number of cancers, including breast, endometrial, and lung. In contrast, 76Br (t1/2 = 16.2 hr) has limited therapeutic potential, but instead emits a number of positrons and high-intensity discrete gamma rays in its electron capture decay. This decay radiation makes it useful as a diagnostic isotope, in particular as a radiotracer for slow biological processes, including as a tracer for mAbs. However, due to its similar chemical properties to the radiohalogens 211At and 77Br, 76Br may be used as a PET / SPECT diagnostic surrogate for these therapeutic radionuclides. This is particularly valuable for 77Br, which can be used to form a theranostic pair with 76Br, exploiting their nearly-identical chemical properties to deliver a mixture of 76Br- and 77Br-labeled targeting vectors for simultaneous, real-time dose delivery and verification. However, as a chemical analogue for both radiohalogens, 76Br may be used to facilitate biodistribution and uptake studies of both 211At and 77Br, to help with preclinical planning of dose prescriptions. Mention low uptake of radiobromines in the thyroid, to spare thyroid (big issue with 125I Auger therapy)?



Figure 2: Summary figure from Theresa, still under development
The four work packages for this project will result in the delivery of capabilities for routine production for three labelled medical radionuclides: 211At, 77Br, and 76Br. This project will start with the basic nuclear physics thin-target measurements for each production reaction, which will be used in designing production targets for large-scale activity production. Following production using these targets, radiochemical workups will purify the desired 211At, 77Br, and 76Br reaction products, and then couple the purified radionuclides to appropriate targeting vectors. Thus, the project will cover a scope leading from basic physics measurements, all the way up to delivering routine quantities of labelled radionuclides, ready to be used in early pre-clinical studies. The synthesis of the chemistry, nuclear physics, and novel target design skills necessary to complete this bench-to-preclinical production is a requirement for developing and bringing a new medical radionuclide to market. This is often an obstacle, as many active research groups focus on a particular stage of this “pipeline,” but without targeted collaboration or integrated campaigns for complete production development, many novel radionuclides remain stuck in this development pipeline.

Due to the breadth of chemistry, physics, and biomedical science which this spans, it is clear that this project is inherently interdisciplinary. As such, the focus will be on collaboratively building routine production capabilities at Oslo for these three new radionuclides. Individually, many of the challenges involved in this work have been addressed for the cases of other radioisotopes in the previous work of both myself and the Oslo group. While novel contributions will be made to advance the state of the technical art in the different work packages along the way, this project will primarily focus on the synthesis of these different disciplines and past experience to create new production capabilities.  Assuming this project is successful, not only will I have gained the necessary skills for the next step in my professional career advancement, but Oslo will be capable of routinely producing the medical radionuclides 211At, 77Br, and 76Br in purified, labelled form. These will be able to be utilized for preclinical studies in other groups at the University of Oslo, as well as other academic and research laboratories throughout Norway and Europe, to help aid in their development for clinical use in treating a variety of cancers plaguing humankind. 





% 
% 
% 


In addition, this work seeks to outline many of the small systematic issues which can be unwittingly introduced into such measurements even with careful experimental design, and the methods developed to deal with them.
Nearly all of the issues presented in this work stem from the use of Kapton tape to encapsulate activation foils and prevent dispersible contamination.
While the issues have been identified and accounted for in the analysis described here, they serve as a cautionary note to future stacked-target cross section measurements.
Finally, this measurement provides some commentary on the importance and selection of monitor reactions, and how \ce{^{93}Nb}(p,4n)\ce{^{90}Mo} fits this perfectly in the intermediate-energy region.
The success of \ce{^{90}Mo} as a monitor reaction product is mainly due to it avoiding the co-production and contamination issues that several of the current monitor standards (namely, Al, Ti, and Ni) are plagued with.


