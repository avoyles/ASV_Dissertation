\chapter{Introduction}


\epigraph{\textit{The story so far:\\In the beginning the Universe was created.\\This has made a lot of people very angry and been widely regarded as a bad move.}}{Douglas Adams \\ The Restaurant at the End of the Universe}

% \Capinsert[4]{\textbf{B}}{ovinely} invasive brag; cerulean forebearance.
% Washable an acre. To canned, silence in foreign.
% Be a popularly. A as midnight transcript alike.
% To by recollection bleeding. That calf are infant. In clause.
% Buckaroo loquaciousness?  Aristotelian!
% Masterpiece as devoted. My primal the narcotic. For cine?
% In the glitter. For so talented. Which is confines cocoa accomplished.
% Or obstructive, or purposeful.
% And exposition? Of go. No upstairs do fingering.

\section{Motivation}





% \vspace{1cm}


% The overarching goal of this project is to develop capabilities for bench-to-preclinical production of the novel emerging medical radionuclides 211At, 77Br, and 76Br at the University of Oslo (UiO), which are desired for personalized cancer therapeutic applications and PET/SPECT diagnostic imaging. 
% These chemically complementary medical radionuclides are further desired for their ability to be generated in theranostic pairs, or even triplets, for next-generation combined therapeutic and imaging treatment applications. 
% The lack of widespread access to clinically relevant quantities of these radionuclides has typically the greatest impediment to advancing their widespread use in pre-clinical studies. 
% Building the capability for production of activity in such quantities will thus help to enable more development of these radionuclides for clinical application. 
\Capinsert[4]{\textbf{R}}{ecent} studies suggest that on average, nearly one in three individuals will be diagnosed with cancer during their lifetime. 
Current treatment options, including surgery, conventional cytotoxins, chemotherapy, and external beam radiation therapy, face several obstacles in effectively treating these diseases. 
Long-term survival is especially challenging for aggressive and invasive strains, as well as metastatic and recurrent cancers. 
In these cases, the cure may be worse than the disease itself, as the aggressive treatment approaches used to combat the spread of disease often cause significant side effects through widespread damage to organs and healthy tissues. 

It is clear that this is a fundamental, systemic problem for society, with inherently interdisciplinary approaches required for the development of next-generation solutions for treatment and detection.  
One such emerging approach is that of targeted radionuclide therapy, which utilizes the intravenous delivery of a therapeutic radionuclide coupled with a \enquote{targeting vector} biomolecule, to precisely deliver a radioactive \enquote{payload} to the site of disease. 
Radionuclide therapy offers the potential benefits of both external beam radiotherapy (destruction of cancer cells by radiation-induced DNA damage) and conventional chemotherapy (systemic treatment throughout the body), without the associated side effects both of these methods commonly produce through accidental damage of healthy tissue.  
In the process of radioactive decay, radionuclides deposit the energy of their decay radiation isotropically. 
This allows radionuclides to deliver a therapeutic dose in an approximately spherical volume around the site of each single nuclide, allowing them to kill a small number of surrounding cells, in addition to the directly targeted cell. 
The choice of a particular radionuclide gives the medical team control over the selectivity of this dose range, leading to the potential to \enquote{paint} a tumor with a \enquote{brush} of tunable width. 
Similarly, candidates for targeting vehicles are chosen to systemically seek out cancerous cells throughout the body, thereby selectively delivering a dose only to the site of disease, sparing healthy tissue and organs throughout the body. 
More importantly, this allows the radionuclides to treat not only any primary tumor sites, but any other undetected metastases which may have spread throughout the body. 
Additionally, instead of a therapeutic radionuclide, one which emits either positrons or a single gamma-ray may be attached to the targeting vector, to detect the presence of cancerous cells through conventional PET or SPECT diagnostic imaging modalities. 
Vitally, this combination of radionuclides and targeting vectors is inherently modular in nature --- for a given radionuclide, different vectors may be coupled to it, based on where the radionuclide is desired to be selectively delivered. 
Conversely, once a targeting vector is established, different radionuclide payloads can be attached to it, based on the range of dose desired for delivery, or for imaging instead.

The promise of these methodologies seeks to shift the paradigm of modern cancer diagnosis and treatment, especially when used in combination. 
The future of nuclear medicine would appear to be  personalized medicine --- targeted radionuclide therapy to spare healthy tissue \cite{Mulford2005,Qaim201731}, and theranostic medicine, which pairs a mixture of an imaging isotope with a therapeutic isotope to provide simultaneous, real-time dose delivery and verification, leading to drastic reductions in received patient dose \cite{Muller2014,Bentzen2005,Srivastava2012}. 
Other variants of theranostic medicine exist, including pre-imaging for treatment planning, or delivery of a single compound with different radioelements for imaging/therapy where the inter-element biodistribution has been validated.  
Relatively few radionuclides possess physical decay characteristics which make them desirable for these applications, so exploratory research is heavily focused on a small number of emerging candidates. Candidate isotopes to meet these needs have been identified based on their chemical and radioactive decay properties \cite{Qaim201731}. 
% As part of my Ph.D. studies 
The work described in this dissertation
% , I have helped lead 
includes a series of 
% campaigns to perform 
targeted, high-priority measurements of thin-target cross sections and thick-target integral yields for many of these emerging novel medical radionuclides. 
These efforts have been motivated by the need to improve existing nuclear data for these valuable production reactions, as well as to ultimately develop capabilities to produce several valuable radionuclides in sufficient quantities to facilitate the production of clinically relevant quantities of radioactivity. 
While this work has contributed to the development of new methods for precision measurements of the production of emerging medical radioisotopes, it has primarily focused on those radionuclides with diagnostic applications. 
However, many of these same methods which I have helped develop may  be applied to  investigations for the production of emerging therapeutic radionuclides as well. 

In selecting therapeutic radionuclides, a vital figure of merit is the linear energy transfer (LET, typically reported in keV/$\mmicro$m) of their decay radiation, which measures the energy deposition per unit length. 
Radionuclides with high-LET radiation produce a high density of ionization events along their trajectories, which cause damage to the integrity of cells and their DNA. 
In addition, LET is inversely proportional to the radius over which this energy is deposited. 
Thus, high-LET radionuclides are prized for therapeutic potential, as their decay radiation produces high cellular lethality over a narrow region, leading to precise delivery of high dose, with minimal dose to surrounding cells. 
Historically, most conventional radionuclide therapy has been reliant upon radioisotopes which decay through $\beta^-$ particle emission, chiefly the radionuclides \ce{^{89}Sr} and \ce{^{131}I}. 
$\beta^-$ particles possess low LET (\textless 0.3\,keV/$\mmicro$m) and long range (100--10,000\,$\mmicro$m) compared to the 10--30\,$\mmicro$m size of most human cells. 
As a result, $\beta^-$ particle therapy has had limited success outside of the treatment of large, solid tumor masses such as prostate cancers. 
This long range makes it difficult to deliver the to deliver the radiation doses needed for irreparable cellular damage to the disease without using high radionuclide concentrations, and in the process, often delivers a high dose to surrounding healthy tissue, as well as the rest of the body. 





% The future of radionuclide therapy appears to be. 

% The future of nuclear medicine would appear to be the paradigm of personalized medicine --- targeted radionuclide therapy to spare healthy tissue , and theranostic medicine, which pairs an imaging isotope with a therapeutic isotope (frequently, of the same element), to provide simultaneous, real-time dose delivery and verification, leading to drastic reductions in prescribed patient dose . 
For novel therapeutic isotopes, active research is  focused on the development of higher-LET isotopes, which generally  fall into two major groups: those which decay by emission of an alpha particle (\enquote{alpha emitters}), as well as those which emit a cascade of Auger electrons in their decay (\enquote{Auger emitters}).  
Many alpha emitters belong to the actinide series and other heavy elements, and possess long decay chains. 
This radiochemical behavior has made handling of many therapeutic alpha emitters challenging, which historically has been an impediment to production and pre-clinical development studies \cite{Mulford2005}. 
However, their decay properties give them high therapeutic efficiency --- the 5--10\,MeV alpha particles commonly emitted possess an LET of 80--100\,keV/$\mmicro$m and a range of 40--100\,$\mmicro$m \cite{Kassis2008}. 
For decay chains involving multiple alpha particle emissions, subsequent decays tend to be extremely short-lived, localizing the dose of the several alpha particles emitted. 
This makes alpha particles extremely lethal to human tissue and their short range (on the order of a single cell) significantly reduces the dose delivered to surrounding healthy tissue \cite{Zalutsky2008,Zalutsky2007}. 
However, in such alpha decay chains, the products along the decay chain tend to diffuse further and further away from the site of the original localized parent isotope, due to degradation of the targeting vector by the emitted alpha particles.
These favorable decay properties, along with the cellular lethality of  alpha particles, has made alpha emitters attractive for a number of therapeutic applications, including the treatment of ovarian and gynaecological cancers, as well as glioblastoma and other recurrent brain cancers \cite{Couturier2005,Zalutsky01102001}. 
Many emerging alpha emitter candidates are also prized for their ability to be easily coupled to monoclonal antibodies  as a targeting vector for selective delivery of the radionuclide.


Most Auger emitters   undergo electron capture decay, leading to the emission of a cascade of low-energy (10\,eV--10\,keV) Auger electrons and Coster-Kronig electrons \cite{adelstein1993merrill,Falzone2012}. 
Such electrons possess an LET of 5--25\,keV/$\mmicro$m, which corresponds to a range of 2--500\,nm \cite{Kassis2008}. 
In addition, due to the electron vacancy cascade mechanism behind their emission, most Auger emitters release between 5--20 electrons in the span of a few femtoseconds following the decay of a single radionuclide, leading to a massive accumulated dose in a volume comparable to the nucleus of a single cell \cite{Pomplun1987}. 
The magnitude of this energy transfer makes Auger emitters suitable for directly inducing double-strand breaks in the DNA of a targeted cancerous cell, from which it is nearly impossible for the cell to repair itself in a way which permits it to survive and divide.  
While the range of Auger electrons prevents them from depositing dose into more than a few nm$^3$ (the few immediately surrounding healthy cells), this extreme dose localization has created challenges in matching radionuclides with suitable targeting vectors \cite{Stepanek1996}.
As a result, the radionuclide must be localized within a cancerous cell for maximum cytotoxicity, leading to delivery approaches which  couple the radionuclide to a biomolecule capable of penetrating the cellular membrane.  
This criterion leads to a selection of radionuclides whose  lifetime permits uptake most commonly by either labeling of various targeted proteins, or direct incorporation into cellular DNA by radionuclide-labeled nucleosides \cite{Kassis1982,Kassis2003}. 
The specificity and high lethality of these electron cascades have made many Auger emitters attractive candidates for the treatment of a number of cancers, including breast, endometrial, and lung.



% Candidate isotopes to meet the needs for both targeted alpha and Auger therapies have been identified based on their chemical and radioactive decay properties. 
% One of the most promising alpha emitters currently being developed is that of the radiohalogen 211At (t1/2=7.21 hr). Every 211At decay leads to the emission of either a 5.87 MeV alpha particle (through the 41.8\% α decay branch of 211At), or a 7.45 MeV alpha particle (through every short-lived t1/2=0.516 s α decay in 211Po, fed by the 58.2\% electron capture branch of 211At), as seen in Figure 1. These alpha particles have an average LET of 97 keV/μm, and range of 50–70 μm in tissue. In decay chains with multiple alpha emissions, the products along the decay chain tend to diffuse further and further away from the site of the original localized parent isotope, due to degradation of the targeting vector by the emitted alpha particles. However, for the simple decay scheme of 211At, little diffusion of the 211Po is observed. This is because 211Po decays rapidly enough following electron capture decay of 211At that the 211Po decays before it has a chance to diffuse far away from the site of the targeted disease. These favourable decay properties, along with the cellular lethality of both alpha particles, has made 211At attractive for a number of therapeutic applications, including the treatment of ovarian and gynaecological cancers, as well as glioblastoma and other recurrent brain cancers. 211At may also be easily coupled to monoclonal antibodies (mAb) as a targeting vector for selective delivery of the radionuclide.

% Another pair of promising radiohalogens are the bromine isotopes 76Br and 77Br. 77Br (t1/2 = 57.04 hr) is an emerging Auger therapy agent, which produces a cascade of approximately, on average, seven low-energy electrons with an average LET of 14 keV/μm: 9.67 and 1.32 keV Auger electrons (with ranges of approximately 3.1 and 0.9 μm in tissue, respectively), and several 20–80 eV Coster–Kronig electrons, with ranges of 1–7 μm in tissue. Like any Auger emitter, the radionuclide must be localized within a cancerous cell for maximum cytotoxicity, so it should be coupled to a biomolecule capable of penetrating the cellular membrane.  Its lifetime permits uptake of targeted 77Br by either labelling of various targeted proteins, or direct incorporation into cellular DNA by radiobromine-labelled nucleosides. The specificity and high lethality of its electron cascade has made 77Br an attractive candidate for the treatment of a number of cancers, including breast, endometrial, and lung. 

However, the development of such novel therapeutic radionuclides is of limited use without parallel advancements in diagnostic applications.
At present, the  medical radionuclides \ce{^{99m}Tc}, \ce{^{18}F}, and \ce{^{68}Ga} make up the backbone of diagnostic nuclear medicine.
However, the usefulness of diagnostic radionuclides is limited to applications where biological uptake permits sufficient detection statistics within patient dose thresholds, and the radiological half-life of the imaging agent is complementary to its biological half-life. 
As a result, the development of a range of new options for diagnostic radionuclides makes a wider range of organs and biological processes accessible to imaging.
This same development effort can be employed for facilitate  non-invasive imaging of model, living systems to rapidly assay the \emph{in vivo} biodistribution of therapeutic radionuclides chemically coupled to biological targeting vectors, necessary in developing therapeutic radiopharmaceuticals.
For these applications, positron emission tomography (PET) imaging is the unquestioned standard, with established quantitative capability for assay in scales as low as  nanomolar concentrations of diagnostic radionuclides. 
The biodistribution signals of these  labeled compounds may be coupled to conventional three-dimensional tomography (CT, MRI), to produce time-dependent uptake studies in anatomical models.
Such studies are noninvasive and minimally perturb living subjects, making these combined imaging modalities one of the most useful tools for pharmacokinetics studies in developing new radiopharmaceuticals.
In helping to develop novel PET isotopes, a current trend is the pursuit of radionuclides with a low Q-value for  $\beta^+$/$\epsilon$ decay.
Since the finite range of an emitted positron before it annihilates is one of the fundamental limits of spatial resolution in PET imaging, radionuclides with a low $Q_\beta$ will produce short-range positrons  \cite{bushberg2011essential}.
However, one of the most useful considerations for developing novel PET radionuclides is the option for a wide range of lifetimes, to target a range of biological processes.
Longer-lived PET isotopes are useful as a  radiotracer for slow biological processes, such as neurological systems, immune studies, and \emph{in vivo} tracking of  monoclonal antibodies.
When considering theranostic applications, these PET isotopes are well-suited for pairing with an Auger-therapeutic agent, as they often rely on slower mechanisms for cellular uptake, including integration into targeted DNA.
Conversely, short-lived PET isotopes are preferred for rapid biological processes such as metabolic studies, or for pairing with a complementary short-lived therapeutic radionuclide.
In considering such theranostic applications, it is preferable to combine together a therapeutic and diagnostic radionuclde from the same chemical group, exploiting their nearly-identical chemical properties to deliver with the same biological uptake a mixture of labeled targeting vectors for simultaneous, real-time dose delivery and verification. 






% In contrast, 76Br (t1/2 = 16.2 hr) has limited therapeutic potential, but instead emits a number of positrons and high-intensity discrete gamma rays in its electron capture decay. This decay radiation makes it useful as a diagnostic isotope, in particular as a radiotracer for slow biological processes, including as a tracer for mAbs. However, due to its similar chemical properties to the radiohalogens 211At and 77Br, 76Br may be used as a PET / SPECT diagnostic surrogate for these therapeutic radionuclides. This is particularly valuable for 77Br, which can be used to form a theranostic pair with 76Br, exploiting their nearly-identical chemical properties to deliver a mixture of 76Br- and 77Br-labeled targeting vectors for simultaneous, real-time dose delivery and verification. However, as a chemical analogue for both radiohalogens, 76Br may be used to facilitate biodistribution and uptake studies of both 211At and 77Br, to help with preclinical planning of dose prescriptions. 






A general  workflow exists in developing capabilities for routine charged-particle production of an emerging medical radionuclide, though the details of each will vary due to the specific challenges involved with each particular radionuclide. 
Any optimal design of a production target for these radionuclides requires well-established knowledge of each of the production cross sections over the energy range being considered. 
In general, the first step of production development begins with a series of low-activity thin-target nuclear activation experiments, utilizing 
% a cyclotron or linac 
an accelerator to measure the production cross section 
% (in mb) of 
for each radionuclide, through observation of decay gamma-rays using a high-purity germanium (HPGe) detector.
% to quantify the activities produced in each activation experiment. 
% The development of the methodology and analytical process for measurements of this nature is one of my areas of expertise, forming the basis of my PhD research and publication history 2. 
% These cross sections will be used in designing the production targets used to produce the larger quantities of each radionuclide for the later purification and labeling work. 
The data from these measurements will be used to design  production targets for each radionuclide, determining the beam energy range which maximizes yield of each radionuclide, while minimizing contamination from other unwanted co-produced isotopes of the product radionuclides. 
These contaminant isotopes serve to lower the specific activity of the final labeled radionuclides, as they cannot be feasibly chemically separated, and serve to deliver unnecessary extra dose to the patients. 
Thus, minimizing their production is the most efficient method of maximizing the use of optimum production targets.

Utilizing this data, production targets will be designed that maximize
% and tuned to match the beam energy for maximum high-purity radionuclide production. 
% These will be activated using the accelerated ion beams, to measure 
the thick-target integral yields (in mCi/$\mmicro$A) to produce the 
% for each target, in the process of producing the 
mCi-scale activities of each radionuclide needed for the purification and labeling work tasks. 
% Ultimately, this quantity (along with the radioisotopic purity of the product) is the desired figure-of-merit in the commercial sector for production.
Following  production, the desired radionuclide products will be recovered from their  targets and purified, generally  using a combination of radiochemical and dry distillation methodologies \cite{Lindegren2001}. 
% These techniques have been explored at OCL for similar activities in the past, but this would be adapted to the specific challenges of these particular radionuclides. 
% This work (along with the thick-target yields) is then disseminated, as the combination of target designs and recovery workflow would provide the foundation for exploitation of these results through commercial production target design. 
Finally,  the purified products are  coupled to appropriate targeting vectors, forming a  \enquote{packaged} batch of each labeled radionuclide, ready to be used in pre-clinical studies for bio-uptake investigations. 



The work detailed in this dissertation focuses on the first step,   the  measurement of excitation functions for the production of a number of emerging medical radionuclides.
The  development of the methodology and analytical process for such measurements is an essential step in this process, as it provides the fundamental, basic science understanding of the physics involved in these reaction regimes.
% Without this basic nuclear data, the efficient fabrication of a production-scale target is nearly impossible.
The measurements described in this work are intended to provide the first step towards enabling each of these  projects to progress to widespread clinical applications. 
It is my  hope that the methods described in this dissertation will be utilized  
% for pre-clinical studies in other research groups, 
to help aid in the development of new radionuclide applications for clinical use in treating a variety of cancers.


\section{Organization of the Dissertation}

This dissertation is organized in the following way.

% The present dissertation represents the summation of work performed during my graduate studies in the development of new production pathways, monitoring capabilities, and analytical methods and tools for the next generation of medical isotope production.
% As such, it is structured in a coherent narrative describing each of these various efforts in detail, often  through referencing of  of my own previously published material.
% Care has been taken to provide additional discussion for each of these areas, documenting the capabilities and details of all facilities used in carrying out this work.
% The fundamental nuclear data used in all analysis has also be recorded here for posterity, to anticipate and facilitate any future need for renormalization of the data reported herein.
% This work also aims to serve as something of a pedagogical text, describing the experimental methods and analytical techniques employed in these measurements.
% This is primarily done to describe the challenges observed in performing this work, outline the solutions and methodology used to overcome these challenges, and allow this collective experience to serve as a compendium of \enquote{lessons learned} to guide any future, similar experiments.



% An accurate integrated proton current is one of the most important factors in performing high-fidelity cross section measurements.
% At the time of this work, the nondestructive beam current monitors in the LANSCE-IPF beamline had a  resolution of 100 nAh.
% For a low-current irradiation such as this work, where a nominal fluence of 200 nAh is desired, additional fluence sensitivity is thus needed to accurately normalize quantified EoB activities into cross sections.
% To this end, 


Chapter \ref{sec:chapter_ipf}  describes the experimental measurement of the \ce{^{93}Nb}(p,4n)\ce{^{90}Mo} reaction as an intermediate-energy proton monitor.
This was carried out through a stacked-target irradiation of thin niobium, copper, and aluminum foils at LANSCE-IPF.
An accurate integrated beam current is one of the most important factors in performing high-fidelity cross section measurements.
At the time of this work, the nondestructive beam current monitors in the LANSCE-IPF beamline had a  resolution of 100\,nAh.
For a low-current irradiation such as this work, where a nominal fluence of 200\,nAh is desired, additional fluence sensitivity is thus needed to accurately normalize quantified end-of-bombardment activities into cross sections.
Developing  new activation foil-based methods for charged particle beam monitors allows users to also gain valuable information about beam energy and systematics, as well as enable measurement of beam fluence at multiple points within a target stack.
The work described in this chapter is the first step in an effort to characterize this reaction as a robust and reliable, contamination-free monitor reaction channel for 40--200\,MeV.
This work also presents an explanation for evidence of \ce{^{nat}Si}(p,x)\ce{^{22,24}Na} contamination, arising from silicone adhesive in the Kapton tape used to encapsulate monitor foils. 
This contamination is frequently seen in stacked-target activation experiments and has the potential to systematically dampen the magnitude of reported cross sections by as much as 50\%. 
This is discussed as a cautionary note to future stacked-target cross section measurements.
This work was also presented in a peer-reviewed publication in Nuclear Instruments and Methods \cite{Voyles2018a}, as well as several conferences and workshops \footnote{A.S. Voyles, \enquote{Isotope production cross section measurements at the HFNG, LANL-IPF, and LBNL.}  \nth{14}  Nordic Meeting on Nuclear Physics, Longyearbyen, Norway. 24 May 2018.  \url{https://indico.cern.ch/event/686407/contributions/2943775/}}\footnote{A.S. Voyles, \enquote{Medical Isotope Production at Berkeley}. University of Oslo Nuclear Physics Summer School, Oslo, Norway.  19 May 2017.  \url{https://github.com/avoyles/presentations/blob/master/2017-05-19-oslo_summer_school/Voyles_19_May_2017_Oslo_Summer_School.pdf}}\footnote{A.S. Voyles, \enquote{Experimental Activities in Berkeley}. US National Nuclear Data Week  (CSEWG), Upton, NY. 14 November 2016.  \url{https://indico.bnl.gov/event/1743/contributions/3189/}}.





Chapter \ref{sec:chapter_hfng}  describes the experimental measurement of the \ce{^{64}Zn}, \ce{^{47}Ti}(n,p) cross sections.
This was carried out at the recently-commissioned UC Berkeley High Flux Neutron Generator, a compact DD neutron generator designed  for geochronology measurements.
This work was motivated by the production of  \ce{^{64}Cu} and \ce{^{47}Sc}, a pair of  emerging medical radionuclides prized in particular for their capacity for theranostic applications. 
Notably, the work presented in this chapter was the first scientific measurement to be carried out in this new research facility, and served to characterize the potential role of compact neutron generators for medical isotope production.
% generator for future neutron production experiments.
% This chapter is important to the narrative of the overall dissertation, as it presents compact DD/DT neutron generators as a viable and novel production pathway for medical radionuclides.
% Conventional (n,x) isotope production is typically performed in thermal nuclear reactors, but suffer from low yields and high radioisotopic contamination. 
% The potential for high-specific activity production and easy deployment, due to compact size and lack of dependence on special nuclear material, allows  DD/DT neutron generators to stand poised as a novel paradigm for high-specific activity isotope production.
% However, an obstacle to wider utilization of  such generators  is the paucity of well-characterized nuclear data for the production of isotopes via (n,p) and (n,$\alpha$) channels, which this work seeks to address.
This work was also presented in a peer-reviewed publication in Nuclear Instruments and Methods \cite{Voyles2017}, as well as several conferences and workshops \footnotemark[1] \footnotemark[2] \footnotemark[3]  \footnote{A.S. Voyles, \enquote{\ce{^{64}Cu} and \ce{^{47}Sc} (n,p) Cross-Section Measurements for Medical Radionuclide Production.} \nth{16} International Workshop on Targetry and Target Chemistry, Santa Fe, NM. 30 August 2016.\\ \url{https://slideslive.com/38898186/64cu-and-47scnp-crosssection-measurements-for-medical-}\\ \url{radionuclide-production}}.




Chapter \ref{sec:chapter_fe}  describes a measurement of the excitation function for production of the  \ce{^{nat}Fe}(p,x)\ce{^{51,52m,52g}Mn} novel PET isotopes, as part of the initial set-up of a new facility and capability for stacked-target cross section measurements.
As the first experiment in blazing a path towards a complementary sister facility to LANSCE-IPF, it is important to note that this chapter focuses on the experimental description and capabilities, with results forthcoming.
This was carried out through a stacked-target irradiation of thin iron, copper, and titanium foils at the Lawrence Berkeley National Laboratory's 88-Inch Cyclotron.
These radionuclides show great promise for a variety of medical applications, but the medical community has been unable to pursue pre-clinical and clinical development due to the lack of well-established production pathways.
This chapter focuses on describing the experimental methods and analysis used for this measurement, and illustrates the importance of accurate knowledge of target composition.
One cross-cutting outcome from this work has been an increased appreciation for the energy lost in the acrylic adhesive on the Kapton tape used to contain the individual stacked targets in these measurements
and the
% While this might seem obvious, the contributions to the slowing of the beam due to the adhesive has often been neglected in much work performed to date. 
% While this plays a limited role at high beam energies, it becomes 
increasingly important role it plays for proton energies below 25\,MeV. 
This work was also presented in several conferences and workshops \footnotemark[1] \footnote{A.S. Voyles, \enquote{Spin Distribution of Excited Nuclear States in $^{\text{nat}}$Fe(p,$\alpha$n).} \nth{6} Workshop on Nuclear Level Density and Gamma Strength, Oslo, Norway. 08 May 2017. \url{http://tid.uio.no/workshop2017/talks/OsloWS17_Voyles.pdf}}.





Finally, Appendix A  contains the various MCNP6 input files used in the analysis of the work presented in this dissertation. 
% They are provided for the purposes of both documentation and reproducibility, that anyone with a license for the code (version MCNP-6.1) might be able to run them.
These models are  used in the analysis of the work in this dissertation for the primary purpose of determining the energy distributions for each irradiation scenario, using the rigorous particle transport methods of the MCNP code.
% Since the energy assignment, and thus, particle fluence, for each of the experiments presented here are based upon these transport models, they represent a systematic factor in the magnitude of all cross sections reported in this work.
By providing the transport models used, these input files allow for the renormalization of the reported cross sections, in the event of an error in the model inputs.


