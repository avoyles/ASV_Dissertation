% (This file is included by thesis.tex; you do not latex it by itself.)

\begin{abstract}

% The text of the abstract goes here.  If you need to use a \section
% command you will need to use \section*, \subsection*, etc. so that
% you don't get any numbering.  You probably won't be using any of
% these commands in the abstract anyway.

The future of nuclear medicine would appear to be the paradigm of personalized medicine --- targeted radionuclide therapy to spare healthy tissue, and theranostic medicine, which pairs an imaging isotope with a therapeutic isotope to provide simultaneous, real-time dose delivery and verification, leading to drastic reductions in prescribed patient dose.
Candidate isotopes to meet these needs have been identified based on their chemical and radioactive decay properties. 
As part this dissertation, I have lead a series of  targeted, high-priority measurements of thin-target cross section, as part of a larger campaign to address deficiencies in cross-cutting nuclear data needs. 
These studies will serve to facilitate the production of pre-clinical quantities of radioactivity for emerging and novel medical radionuclides. 

This dissertation details a series of three  experiments, focusing on production pathways for the radionuclides \ce{^{90}Mo}, \ce{^{51}Mn}, \ce{^{52m}Mn}, \ce{^{52g}Mn}, \ce{^{64}Cu}, and \ce{^{47}Sc}.
Each of these experiments has been designed as part of efforts to measure production cross sections for emerging medical radionuclides and develop new methods for the monitoring of charged-particle beams.
The discussion focuses on describing the experimental methods and analysis used for this measurement, and illustrates the importance of accurate knowledge of target composition.
The experimental measurement of the \ce{^{93}Nb}(p,4n)\ce{^{90}Mo} reaction was motivated by its use as an intermediate-energy proton monitor, and was  carried out through a stacked-target irradiation of thin niobium, copper, and aluminum foils at LANSCE-IPF.
% An accurate integrated beam current is one of the most important factors in performing high-fidelity cross section measurements.
% At the time of this work, the nondestructive beam current monitors in the LANSCE-IPF beamline had a  resolution of 100\,nAh.
% For a low-current irradiation such as this work, where a nominal fluence of 200\,nAh is desired, additional fluence sensitivity is thus needed to accurately normalize quantified end-of-bombardment activities into cross sections.
% Developing  new activation foil-based methods for charged particle beam monitors allows users to also gain valuable information about beam energy and systematics, as well as enable measurement of beam fluence at multiple points within a target stack.
The work described in this chapter is the first step in an effort to characterize this reaction as a robust and reliable, contamination-free monitor reaction channel for 40--200\,MeV protons.
The  measurement of the excitation function for   \ce{^{nat}Fe}(p,x)\ce{^{51,52m,52g}Mn} was motivated by the production of these novel PET isotopes for a variety of diagnostic and imaging modalities.
This was carried out through a stacked-target irradiation of thin iron, copper, and titanium foils at the Lawrence Berkeley National Laboratory's 88-Inch Cyclotron.
% These radionuclides show great promise for a variety of medical applications, but the medical community has been unable to pursue pre-clinical and clinical development due to the lack of well-established production pathways.
The measurement of the \ce{^{64}Zn}, \ce{^{47}Ti}(n,p) cross sections  was carried out at the recently-commissioned UC Berkeley High Flux Neutron Generator, a compact DD neutron generator designed originally for geochronology measurements.
This work was motivated by the production of  \ce{^{64}Cu} and \ce{^{47}Sc}, a pair of  emerging medical radionuclides prized in particular for their capacity for theranostic applications. 
% Notably, the work presented in this chapter was the first scientific measurement to be carried out in this new research facility, and served to characterize the generator for future neutron production experiments.
% This chapter is important to the narrative of the overall dissertation, as it presents compact DD/DT neutron generators as a viable and novel production pathway for medical radionuclides.
% Conventional (n,x) isotope production is typically performed in thermal nuclear reactors, but suffer from low yields and high radioisotopic contamination. 
% The potential for high-specific activity production and easy deployment, due to compact size and lack of dependence on special nuclear material, allows  DD/DT neutron generators to stand poised as a novel paradigm for high-specific activity isotope production.
% However, an obstacle to wider utilization of  such generators  is the paucity of well-characterized nuclear data for the production of isotopes via (n,p) and (n,$\alpha$) channels, which this work seeks to address.
The work described here may help to enable exciting new campaigns of investigation in basic science, disease biology research, and nuclear medicine.


This dissertation serves as a pedagogical example to those who follow of how the assortment of unexpected difficulties in precision nuclear data measurements can make \enquote{simple} experiments not so simple, after all.
One cross-cutting outcome from this work has been an increased appreciation for the role played by the acrylic adhesive on the Kapton tape used to contain the individual stacked targets in these measurements. 
This work  presents an explanation for evidence of \ce{^{nat}Si}(p,x)\ce{^{22,24}Na} contamination, arising from silicone adhesive in the Kapton tape. 
This contamination is frequently seen in stacked-target activation experiments and has the potential to systematically dampen the magnitude of reported cross sections by as much as 50\%. 
This is discussed as a cautionary note to future stacked-target cross section measurements.
% This chapter focuses on describing the experimental methods and analysis used for this measurement, and illustrates the importance of accurate knowledge of target composition.
In addition,  contributions to the slowing of charged particle beams due to the adhesive have often been neglected in much work performed to date. 
While this plays a limited role at high beam energies, it becomes increasingly important for proton energies below 25\,MeV. 






\end{abstract}
