\chapter{Conclusions}

\Capinsert[4]{\textbf{B}}{ovinely} invasive brag; gait grew Fuji Budweiser penchant walkover pus hafnium
financial Galway and punitive Mekong convict defect dill, opinionate
leprosy and grandiloquent?  Compulsory Rosa Olin
% Jackson\cite{waveshaping} and pediatric Jan.  Serviceman, endow buoy
apparatus.


% In addition, this work seeks to outline many of the small systematic issues which can be unwittingly introduced into such measurements even with careful experimental design, and the methods developed to deal with them.
% Nearly all of the issues presented in this work stem from the use of Kapton tape to encapsulate activation foils and prevent dispersible contamination.
% While the issues have been identified and accounted for in the analysis described here, they serve as a cautionary note to future stacked-target cross section measurements.
% Finally, this measurement provides some commentary on the importance and selection of monitor reactions, and how \ce{^{93}Nb}(p,4n)\ce{^{90}Mo} fits this perfectly in the intermediate-energy region.
% The success of \ce{^{90}Mo} as a monitor reaction product is mainly due to it avoiding the co-production and contamination issues that several of the current monitor standards (namely, Al, Ti, and Ni) are plagued with.
% 
% 
% 
% 
% mention toolkit of options
% 
% 
% 
% give description of pathway to pre-clnical work 
% 
% 
% The four work packages for this project will result in the delivery of capabilities for routine production for three labelled medical radionuclides: 211At, 77Br, and 76Br. This project will start with the basic nuclear physics thin-target measurements for each production reaction, which will be used in designing production targets for large-scale activity production. Following production using these targets, radiochemical workups will purify the desired 211At, 77Br, and 76Br reaction products, and then couple the purified radionuclides to appropriate targeting vectors. Thus, the project will cover a scope leading from basic physics measurements, all the way up to delivering routine quantities of labelled radionuclides, ready to be used in early pre-clinical studies. The synthesis of the chemistry, nuclear physics, and novel target design skills necessary to complete this bench-to-preclinical production is a requirement for developing and bringing a new medical radionuclide to market. This is often an obstacle, as many active research groups focus on a particular stage of this “pipeline,” but without targeted collaboration or integrated campaigns for complete production development, many novel radionuclides remain stuck in this development pipeline.
% 
% Due to the breadth of chemistry, physics, and biomedical science which this spans, it is clear that this project is inherently interdisciplinary. As such, the focus will be on collaboratively building routine production capabilities at Oslo for these three new radionuclides. Individually, many of the challenges involved in this work have been addressed for the cases of other radioisotopes in the previous work of both myself and the Oslo group. While novel contributions will be made to advance the state of the technical art in the different work packages along the way, this project will primarily focus on the synthesis of these different disciplines and past experience to create new production capabilities.  Assuming this project is successful, not only will I have gained the necessary skills for the next step in my professional career advancement, but Oslo will be capable of routinely producing the medical radionuclides 211At, 77Br, and 76Br in purified, labelled form. These will be able to be utilized for preclinical studies in other groups at the University of Oslo, as well as other academic and research laboratories throughout Norway and Europe, to help aid in their development for clinical use in treating a variety of cancers plaguing humankind. 


