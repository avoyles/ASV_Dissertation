\chapter{Conclusions and Outlook}

\Capinsert[4]{\textbf{I}}{n this} work, we have explored a variety of pathways for detailed measurements of the excitation functions of medical radionuclides. 
These have all been selected to provide basic science advances in enabling production through  existing pathways for medical isotope production, as well for promising potential future pathways:
\begin{itemize}
 \item Characterization of the \ce{^{93}Nb}(p,4n)\ce{^{90}Mo} reaction as an intermediate-energy proton monitor;
 \item Production of the  \ce{^{51,52m,52g}Mn} novel PET isotopes through low-energy \ce{^{nat}Fe}(p,x) reactions;
 \item Production of  \ce{^{64}Cu} and \ce{^{47}Sc} via \ce{^{64}Zn},\ce{^{47}Ti}(n,p) reactions using a compact DD generator.
\end{itemize}


In addition, this work seeks to outline many of the small systematic issues which can be unwittingly introduced into such measurements even with careful experimental design, and the methods developed to deal with them.
Nearly all of the issues presented in this work stem from the use of Kapton tape to encapsulate activation foils and prevent dispersible contamination.
One cross-cutting outcome from this work has an increased appreciation for the role played by the silicone adhesive on this tape used to contain the individual stacked targets in these measurements. 
While this might seem obvious, the contributions to the slowing of the beam due to the adhesive has often been neglected in much work performed to date. 
While this plays a limited role at high beam energies, it becomes increasingly important for proton energies below 25\,MeV. 
This work also presents an explanation for evidence of \ce{^{nat}Si}(p,x)\ce{^{22,24}Na} contamination, arising from silicone adhesive in the Kapton tape used to encapsulate monitor foils. 
This contamination is frequently seen in stacked-target activation experiments and has the potential to systematically dampen the magnitude of reported cross sections by as much as 50\%. 
This is a poignant reminder of the importance and selection of monitor reactions, and how \ce{^{93}Nb}(p,4n)\ce{^{90}Mo} fits this perfectly in the intermediate-energy region.
% This is discussed as a cautionary note to future stacked-target cross section measurements.
While these issues have been identified and accounted for in the analysis described here, they serve as a cautionary note to future stacked-target cross section measurements.
In addition to the novelty of advancing basic science, the measurement of these reactions provides an example of the poor current state of modeling for proton-induced nuclear reactions in the pre-equilibrium region.
The nuclear data measured here provides a novel contribution in the fact that it may be able to be used as input parameters to tune and improve reaction modeling in this mass region by providing insight into the pre-equilibrium reaction mechanisms that play such an important role in this energy region.





This project has been focused almost completely on  the basic nuclear physics thin-target measurements for each production reaction.
The next step in continuing this work will be to repeat these measurements, extending the energy range explored in each of these experiments, and filling in the energy points in between those measured here for each excitation function. 
With a more well-characterized excitation function, these data  will be used in designing production targets for large-scale (mCi-scale) activity production. 
Following production using these targets, radiochemical workup will purify the desired reaction products, and then couple the purified radionuclides to appropriate targeting vectors. 
Thus, the project will cover a scope leading from basic physics measurements, all the way up to delivering routine quantities of labelled radionuclides, ready to be used in early pre-clinical studies. 
The synthesis of the chemistry, nuclear physics, and novel target design skills necessary to complete this bench-to-preclinical production is a requirement for developing and bringing a new medical radionuclide to market.
Without targeted collaboration or integrated campaigns for complete production development, many novel radionuclides remain stuck in this development pipeline.



Due to the breadth of chemistry, physics, and biomedical science which this development process requires, it is clear that this project is inherently interdisciplinary.
Part of doing the itinerant work of science is often leaving good work behind undone, as a retained connection to a former institution, and helping those that follow  to improve upon the work you've done. 
Moving beyond the  work in the present dissertation would likely have excessively broadened its scope.
However, the work described here may help to enable exciting new campaigns of investigation in basic science, disease biology research, and nuclear medicine.
I hope to continue down this pipeline in my next activities, developing the radiochemical skills and expertise necessary for enabling me to help lead a complete, bench-to-pre-clinical campaign of novel medical radionuclide development.
The production pathways described in this work  will hopefully result in the development of capabilities for routine production for the labeled medical radionuclides  \ce{^{51,52m,52g}Mn},   \ce{^{64}Cu}, and \ce{^{47}Sc}. 
It is thus my fervent hope that these will be able to be utilized for preclinical studies in other  academic and research laboratories, to help aid in their development for clinical use in treating a variety of diseases plaguing humankind. 


% 
% 
% 
% 
% mention toolkit of options
% 
% 
% 
% give description of pathway to pre-clnical work 
% 
% 

% 
% Due to the breadth of chemistry, physics, and biomedical science which this spans, it is clear that this project is inherently interdisciplinary. As such, the focus will be on collaboratively building routine production capabilities at Oslo for these three new radionuclides. Individually, many of the challenges involved in this work have been addressed for the cases of other radioisotopes in the previous work of both myself and the Oslo group. While novel contributions will be made to advance the state of the technical art in the different work packages along the way, this project will primarily focus on the synthesis of these different disciplines and past experience to create new production capabilities.  Assuming this project is successful, not only will I have gained the necessary skills for the next step in my professional career advancement, but Oslo will be capable of routinely producing the medical radionuclides 211At, 77Br, and 76Br in purified, labelled form. These will be able to be utilized for preclinical studies in other groups at the University of Oslo, as well as other academic and research laboratories throughout Norway and Europe, to help aid in their development for clinical use in treating a variety of cancers plaguing humankind. 


