% \usepackage{siunitx}

\chapter{Excitation functions for (p,x) reactions of niobium (E\texorpdfstring{$_{\text{p}}$=40--90\,}{Ep = 40--90 }MeV):  development of the \texorpdfstring{\ce{^{93}Nb}(p,4n)\ce{^{90}Mo}}{93Nb(p,4n)90Mo} reaction as an intermediate-energy proton monitor}\label{sec:chapter_ipf}
% Shortened title for header
\chaptermark{Development of the \ce{^{93}Nb}(p,4n)\ce{^{90}Mo} reaction as an intermediate-energy proton monitor}


\Capinsert[4]{\textbf{B}}{ovinely} invasive brag; cerulean forebearance.
Washable an acre. To canned, silence in foreign.
Be a popularly. A as midnight transcript alike.
To by recollection bleeding. That calf are infant. In clause.
Buckaroo loquaciousness?  Aristotelian!
Masterpiece as devoted. My primal the narcotic. For cine?
In the glitter. For so talented. Which is confines cocoa accomplished.
Or obstructive, or purposeful.
And exposition? Of go. No upstairs do fingering.

%  Move these first two commented sentences into PhD thesis.
% 
% % % 
% An accurate integrated proton current is one of the most important factors in performing high-fidelity cross section measurements.
% At the time of this work, the nondestructive beam current monitors in the LANSCE-IPF beamline had a  resolution of 100 nAh.
% For a low-current irradiation such as this work, where a nominal fluence of 200 nAh is desired, additional fluence sensitivity is thus needed to accurately normalize quantified EoB activities into cross sections.


% Variance minimization techniques were employed  to correct for uncertainties in  the characterization of the stack components, the largest cause of uncertainties in energy and  fluence assignments.

\vspace{1cm}

\noindent \textbf{Relevant Publications:}

\vspace{0.5cm}


\hangindent=\parindent  \textbf{Andrew S. Voyles}, Lee A. Bernstein, Eva R. Birnbaum, Jonathan W. Engle,
Stephen A. Graves, Toshihiko Kawano, Amanda M. Lewis, and Francois M. Nortier, \enquote{Excitation functions for (p,x) reactions of niobium (E$_{\text{p}}$ = 40--90 MeV):  development of the \ce{^{93}Nb}(p,4n)\ce{^{90}Mo} reaction as an intermediate-energy proton monitor.} Nuclear Instruments and Methods in Physics Research Section B: Beam Interactions with Materials and Atoms, 
% vol. 410, pp. 230--239, Nov. 2017.
% Update following acceptance
(Submitted 2018).
% \cite{Voyles2017}

\vspace{0.5cm}


% T.H. Joshi, S. Sangiorgio, V. Mozin, E.B. Norman, P. Sorensen, M. Foxe, G. Bench, A. Bernstein. Design and characterization of a quasi-monoenergetic neutron source. Nuclear Instruments and Methods in Physics Research B (in press). [44]


% Does \mmicro still work?
The text and figures of this paper, of which I was the primary author, are
included in this chapter with the permission of all authors. 
% Additional discussion of the installation of the neutron source at CAMS and problems with the Li-target are included in Appendix A.


\section{Transitory stuff}


Every year, approximately 17 million nuclear medicine procedures (both diagnostic and therapeutic) are performed in the U.S. alone,  which has made incredible strides in improving our ability to detect and treat a variety of life-threatening diseases \cite{Delbeke2011,NSACIsotopesSubcommittee2015}.
The vast majority of the radioisotopes currently used for these procedures are produced in the field's array of low- (E \textless\ 30 MeV / A) and intermediate-energy (30 \textless\ E \textless\ 200 MeV / A) accelerator capabilities, which routinely produce many of the staple medical radionuclides, such as \ce{^{11}C}, \ce{^{18}F}, \ce{^{68}Ge}, \ce{^{82}Rb}, and \ce{^{123}I}, as well as many  
% of the 
non-medical radioisotopes of commercial value, such as  \ce{^{32}Si}, \ce{^{73}As}, \ce{^{95m}Tc}, and  \ce{^{109}Cd} \cite{international2009iaea,schlyer2008cyclotron}. 
% \comment{Should \enquote{intermediate-energy} be less than 200 MeV, or 100 MeV?  I find conflicting descriptions in the literature.\\Lee:  I believe that 60 MeV is considered as the transition point\\ Stephen: In this context, I would make this definition based on what machines in the US are routinely used for medical isotope production - based on this, I agree with 200 MeV. }
The future of nuclear medicine would appear to be the paradigm of personalized medicine --- targeted radionuclide therapy to spare healthy tissue \cite{Mulford2005,Qaim201731}, and theranostic medicine, which pairs an imaging isotope with a therapeutic isotope (frequently, of the same element), to provide simultaneous, real-time dose delivery and verification, leading to drastic reductions in prescribed patient dose \cite{Muller2014,Bentzen2005,Srivastava2012}. 
Other variants of theranostic medicine exist, including pre-imaging for treatment planning, or delivery of a single compound with different radioelements for imaging/therapy where the inter-element biodistribution has been validated.  
Candidate isotopes to meet these needs have been identified based on their chemical and radioactive decay properties \cite{NSACIsotopesSubcommittee2015,Qaim201731,bernstein2015nuclear}, and a series of campaigns are underway to perform targeted, high-priority measurements of thin-target cross sections and thick-target integral yields.
These studies will serve to facilitate the production of pre-clinical quantities of radioactivity.


One significant obstacle to both high-fidelity measurements of production cross sections for emerging medical radionuclides, as well as conventional isotope production, is a lack of well-characterized  monitor reaction standards.  
This is particularly true for higher-energy charged particle beams, where there is currently a paucity of such data. 
Indeed, the development of new monitor reaction standards and the improved evaluation of existing standards is one of the areas of greatest cross-cutting need for nuclear data \cite{bernstein2015nuclear}. 
Charged particle monitor reaction data currently exists for several low-to-intermediate energy charged particle beams (E \textless\ 50 MeV / A), but even the experimental data used for this evaluation are  sparse above approximately 30 MeV / A and  uncertainties in experimental cross sections are large (6--15\%) \cite{gul2001charged}. 
While work is needed to improve upon existing monitor reaction data, the development of new reactions can expand the available range of options for the monitoring of charged particle beams.
This work in particular seeks to improve this range of options by characterizing a new monitor reaction for  proton beams in excess of 40 MeV, for possible use at isotope production facilities such as BLIP, IPF, or iThemba LABS.
% \comment{Stephen: I'm having a hard time following this paragraph. Some of your syntax overlaps significantly with nuclear medicine \& medical physics terminology with different intended meanings. For example 'dosimetry' is typically taken to mean the measurement of absorbed energy per unit mass (J/kg) and associated measurements/specifications. Another example, 'dosimetry standard' is used quite often when referring to NIST air kerma reference standards which are the bedrock of Accredited Dosimetry Calibration Lab (ADCL) measurements. As you are more familiar with classic Nuclear Physics writing, perhaps you better know how this language will be interpreted. }



Activation analysis is one of the most fundamental measurement techniques in experimental nuclear physics, as it is a simple and straightforward method to probe the structure and behavior of nuclear matter,  dating back to the infancy of the field. 
All activation measurements involve the analysis and quantification of decaying radioactive nuclei created through irradiation via ionizing radiation \cite{ehmann1993radiochemistry,krüger1971principles}.
% \comment{Lee:  I think that you might want to mention a connection to BLIP, IPF, iThemba etc. here, and make it clear that you're dealing with protons only, not neutrons.  }
Monitor reactions have  historically been part of such activation experiments, and serve two valuable purposes for charged particle-induced reactions, depending upon the energy regime.  
% \comment{Stephen:  Move much of the remaining intro section into the Discussion, for brevity?}
Between the reaction's energetic threshold  and the end of its compound peak, the magnitude and shape of a monitor reaction's excitation function changes rapidly with increasing energy, making it useful for determining the energy distribution of particles which have traversed a thin irradiated target.
This is particularly the case when comparing  monitor reactions leading to two distinct residual nuclei from the same target, such as the $^\text{nat}$Cu(p,x)\ce{^{62}Zn} and $^\text{nat}$Cu(p,x)\ce{^{63}Zn} reactions \cite{gul2001charged}.
This is extremely valuable, as it allows the screening and minimization of systematic errors based on energy determination, though this sensitivity to energy precludes their reliability as a beam current monitor.  

Moving to the higher energy  of the reaction's pre-equilibrium tail, the excitation function becomes  smooth and generally flat as a function of energy.
In this regime, the monitor reaction offers little-to-no energy sensitivity. 
However,  in the pre-equilibrium regime, monitor reactions become extremely useful for determining the integral beam current. 
While cross section measurements often use external beam current monitors (such as an inductive pickup upstream of a target, or an electrically-isolated target in a Faraday cup), these measure the integrated current incident upon an entire target assembly.
For the case of stacked-target activation experiments, commonly employed to measure cross sections at multiple energies  in a single activation, external beam current monitors can only measure the integral current incident upon the \enquote{front} (upstream) of the target stack.
In these experiments, a series of monitor foils at each energy position allows one to indirectly measure the integral current at each position in the stack, reducing systematic errors in observed cross section magnitude, but with reduced precision compared to direct measurement using a well-characterized suppressed Faraday cup.
% \comment{Stephen:  ``and precision, unfortunately ''.  Update post-discussion.}
Both of these purposes make well-characterized monitor reactions an invaluable asset to any activation experiment. 



In theory, nearly any radioisotope can serve as a reaction monitor, but those desired to be classified as a monitor reaction standard possess several hallmark characteristics.
The primary factor involved in selecting a new monitor is ensuring that the desired radionuclide emit  at least one (preferably multiple, to ensure accurate radionuclide identification) distinct decay gamma-rays which can be used to uniquely identify it during post-activation assay.  
Generally, this means selecting a radionuclide with a number of distinct gamma-rays.
The decay radiation should preferably have high intensities, so that they show up as strong peaks, and minimize the amount of time needed to count the activated target in order to achieve acceptable counting statistics. 


% \comment{Stephen:  These next 2 paragraphs are better suited for the discussion, specifically addressing how this problem affects your particular experiment. }
Care should be taken to avoid cases where two radionuclides which are produced by two different reactions on the same monitor foil lead to states in the same daughter nuclide.  
For example, \ce{^{48}V}  ($t_{1/2}$ = 15.97 d, $\epsilon=100\%$ to \ce{^{48}Ti}) and \ce{^{48}Sc}  ($t_{1/2}$ = 43.67 h, $\beta^-=100\%$ to \ce{^{48}Ti}) can both be formed via $^\text{nat}$Ti(p,x) reactions, yielding the same 983.52 keV transition in \ce{^{48}Ti} \cite{Burrows2006}.
Fortunately, these cases can occasionally be mitigated by either using a difference in half-life between the two feeding pathways to allow one to decay out, or by using a distinct gamma-ray from one of the two isobar nuclei to subtract out the activity associated with it (such as the $E_\gamma=1037.522$ keV, $I_\gamma=97.6\%$ line in the decay of \ce{^{48}Ti}) \cite{Burrows2006}.
However, this approach propagates larger uncertainties into the final activity of the desired monitor nucleus, so in principle it is far preferred to choose a monitor reaction which does not have overlapping gamma-rays from another isobar nucleus.

Another important decay factor to consider is that of the half-life of the desired monitor nucleus.
Ideally, the nucleus has a lifetime which is sufficiently long-lived to ensure that it may be quantified  conveniently and leisurely after end-of-beam without the majority if it decaying away.
In addition, it is preferred that the lifetime be comparable to that of the reaction products being studied. 
For proper quantification, it is also of vital importance that the proposed monitor nucleus have well-characterized decay data.
This includes a precise and well-established half-life,  needed to  correct for decay losses, as well as well-characterized decay gamma-ray intensities.
In practice, the weakest components of decay data are often the gamma-ray intensities, which can routinely have uncertainties of 5\% or more.
Since this uncertainty is propagated in quadrature from the activity of both the monitor reaction and the reaction product being studied, choosing a monitor with a well-established gamma-ray intensity can make a significant reduction in measured cross section uncertainties.


From a targetry  perspective, it is preferable to use a naturally mono-isotopic target that is readily commercially available at an affordable price and is generally chemically inert --- any significant chemical changes during target preparation (significant oxidation, etc) will affect the target's areal density, systematically changing the measured integral current. 
Structurally, the target material should be malleable and supportive to be able to be formed into a thin target.
For charged particle reactions,  energy degradation scales with target areal density,  broadening the energy spectrum downstream of the target.
However, since the monitor reaction yield also scales with target areal density, the use of a target which is too thin may provide insufficient counting statistics during decay spectroscopy.
For reference, a monitor foil of approximately 25 mg/cm$^2$ provides a good compromise, with less than 100 keV degradation for a proton energy of 100 MeV, and less than 200 keV at 40 MeV.
Thickness selection will be subject to the context of an experiment, seeking to maximize thickness without overly perturbing the energy uncertainty of  measurements.
% These are the primary characteristics involved in choosing an appropriate target for a monitor reaction



% \comment{Stephen:  This is also a discussion point. If you're trying to introduce the purpose of this paper, i.e. the establishment of a new monitor reaction, I would try to condense paragraphs 3--9 into two paragraphs: one paragraph explaining what a monitor reaction is used for, and the other describing considerations relevant to the use of a monitor reaction. The rest of this content can be included in the discussion if you feel it is necessary.}
Lastly, and perhaps most importantly for high-energy monitor reaction applications, it is  of utmost importance to choose a reaction channel which cannot be populated via secondary particles incident upon the monitor target.
This is typically mostly a concern for secondary neutrons produced through (z,xn) reactions on upstream targets, degraders, and stack materials, to avoid monitor reactions which can be populated through (n,x) reactions on the target.
Any monitor reaction channel which can be populated by anything other than the primary beam should be avoided, as it is often a laborious task to separate out the fraction of secondary particles contributing to the total activation.  





One  reaction which satisfies these requirements is that of a new, intermediate-energy proton monitor reaction standard based on \ce{^{93}Nb}(p,4n)\ce{^{90}Mo}. 
Niobium is naturally mono-isotopic, readily  available commercially in high purity, is chemically inert, and can easily be rolled down to foils as thin as 1 \mmicro m.  
\ce{^{90}Mo} also has excellent decay properties --- its fairly long-lived lifetime ($\epsilon=100\%, t_{1/2}=5.56 \pm 0.09$ h) allows it to be counted at leisure without fear of the product \ce{^{90}Mo} decaying away excessively between end-of-beam and the start of counting, and it possesses seven strong, distinct gamma lines (notably its 122.370 keV ($I_\gamma = 64 \pm 3\%$) and 257.34 keV ($I_\gamma = 78 \pm 4\%$) lines) which can be used to uniquely and easily   quantify \ce{^{90}Mo} production \cite{Browne1997}. 
In addition,  \ce{^{90}Mo} is completely immune from (n,x) production on  \ce{^{93}Nb}, being produced only via the primary proton beam, and the \ce{^{90}Mo} decay lines can only be observed in its decay, as its daughter, \ce{^{90}Nb}, is also unstable and decays via $\epsilon$ to stable \ce{^{90}Zr}. 
 
The purpose of the present work is to  measure the production of the long-lived radionuclide \ce{^{90}Mo} ($t_{1/2}=5.56 \pm 0.09$ h \cite{Browne1997}) via the $^\text{nat}$Nb(p,x) reaction. 
In addition to the $^\text{nat}$Nb(p,x)\ce{^{90}Mo} measurement, this experiment has also yielded measurements of 37 other (p,x) production cross sections between 40--90 MeV  for a number of additional reaction products, including several emerging radionuclides with medical applications.
These include the non-standard positron emission tomography (PET) agent \ce{^{57}Ni}  ($t_{1/2}=35.60\pm0.06$ h \cite{Bhat1998}), \ce{^{64}Cu} ($t_{1/2}=12.701 \pm 0.002$ h \cite{Singh2007}), \ce{^{86}Y}  ($t_{1/2}=14.74\pm0.02$ h \cite{NEGRET20151}), \ce{^{89}Zr} ($t_{1/2}=78.41\pm0.12$ h \cite{Singh2013}),  \ce{^{90}Nb} ($t_{1/2}=14.60 \pm 0.05$ h \cite{Browne1997}),  and the diagnostic agent \ce{^{82\text{m}}Rb} ($t_{1/2}=6.472\pm0.006$ h \cite{Tuli2003}). 


%   \comment{The following paragraph should only be included if we decide to touch on spin physics in this manuscript, or split it into a follow-on PhysRev C, etc.}
%   
%   \comment{Stephen:  If included in this paper, I would move to the discussion. It's not clear yet in the paper that you have the data required to make these determinations.}

  
In addition to being a potentially highly-valuable beam monitor, the Nb(p,x) reactions offer an opportunity to study the angular momentum deposition via pre-equilibrium reactions and the spin distribution in g$_{9/2}$ subshell nuclei via the observation of isomer-to-ground state ratios.  
Measurements of isomer-to-ground state ratios have been used for over 20 years to probe the spin distribution of excited nuclear states in the A $\approx$ 190 region \cite{PhysRevC.73.034613,PhysRevC.45.1171}.
These include the \ce{^{52\text{m}}Mn} ($t_{1/2}=21.1\pm0.2$ m; J$^\pi=2^+$) to \ce{^{52\text{g}}Mn}  ($t_{1/2}=5.591\pm0.003$ d; J$^\pi=6^+$), \ce{^{58\text{m}}Co} ($t_{1/2}=9.10\pm0.09$ h; J$^\pi=5^+$) to \ce{^{58\text{g}}Co}  ($t_{1/2}=70.86\pm0.06$ d; J$^\pi=2^+$),  \ce{^{85\text{m}}Y} ($t_{1/2}=4.86\pm0.13$ h; J$^\pi=\sfrac{9}{2}^+$) to \ce{^{85\text{g}}Y}  ($t_{1/2}=2.68\pm0.05$ h; J$^\pi=\sfrac{1}{2}^-$),  \ce{^{87\text{m}}Y} ($t_{1/2}=13.37\pm0.03$ h; J$^\pi=\sfrac{9}{2}^+$) to \ce{^{87\text{g}}Y}  ($t_{1/2}=79.8\pm0.3$ h; J$^\pi=\sfrac{1}{2}^-$),  and \ce{^{89\text{m}}Nb} ($t_{1/2}=66\pm2$ m; J$^\pi=\sfrac{1}{2}^-$) to \ce{^{89\text{g}}Nb}  ($t_{1/2}=2.03\pm0.07$ h; J$^\pi=\sfrac{9}{2}^+$)  ratios \cite{Dong2015,Nesaraja2010,Singh2014,Johnson2015,Singh2013}.  
 
 
The measurements described in this paper involve the use of multiple monitor reactions in conjunction with statistical calculations and proton transport simulations to reduce systematic uncertainties in beam energy assignments, leading to some of the first and most precise measurements  for many of the excitation functions reported here. 
By expanding the available set of monitor reaction standards and well-characterized isotope production excitation functions, this work should help optimize medical isotope production modalities, making more options   available for modern medical imaging and cancer therapy.






% %
% 
%  Dump abstract text from HFNG (n,p) paper into this chapter
% 
% % 
\section{Abstract}
\input{../Manuscripts/nb_px_paper/nb_abstract_text}


% % 
% 
%  Dump body text from HFNG (n,p) paper into this chapter
% 
% % 
\input{../Manuscripts/nb_px_paper/nb_body_text}


% % 
% 
%  Dump appendices text from HFNG (n,p) paper into this chapter
% 
% % 
\input{../Manuscripts/nb_px_paper/nb_appendix_text}


\section{Additional information on the analysis}


% Clean this up

add pictures of gafchromic films 



\begin{figure}
    \centering
    \subfloat{
        \centering
%         \includegraphics[width=\columnwidth]{./figures/Capture.PNG}
        \hspace{-5pt}\subfigimg[width=0.5\textwidth]{a)}{./figures/DOC013119-cropped.pdf}{80}
%         \caption{ Decay curve for the isomeric transition of \ce{^{115m}In}.}
         %         \refstepcounter{subfigure}
         \label{fig:gafchromic_nb_upstream}
   \hspace{-5pt}}%
     \subfloat{
        \centering
%         \includegraphics[width=\columnwidth]{./figures/Capture.PNG}
%         \includegraphics[scale=0.6]{./figures/391keV_curve2.png}
        \subfigimg[width=0.5\textwidth]{b)}{./figures/DOC013118-cropped.pdf}{80}
%         \caption{ Decay curve for the isomeric transition of \ce{^{113m}In}.}
         %         \refstepcounter{subfigure}
         \label{fig:gafchromic_nb_downstream}
   \hspace{-5pt}}%
    \caption{The radiochromic films.}
     \label{fig:gafchromic_nb}
\end{figure}

picture of mcnp stack model


\begin{figure}
 \centering
 %trim option's parameter order: left bottom right top
 \includegraphics[trim = 0mm 0mm 2mm 0mm, clip,width=0.75\columnwidth]{./figures/mcnp_vised2.png}
 % mcnp_vised2.PNG.png: 688x443 pixel, 96dpi, 18.21x11.72 cm, bb=0 0 516 332
 \caption{Blah blah blah
%  MCNP6 model of the HFNG target chamber, with reference scale. The co-loaded foils can be seen in the target chamber center.  The ovals indicate the location of water cooling channels.
}
 \label{fig:ipf_vised}
\end{figure}



% While these TENDL cross sections simply provide approximate \ce{^{nat}Si}(p,x)\ce{^{22,24}Na} yields,  t
There are several important  conclusions to be drawn from this simple estimate using the TENDL   \ce{^{nat}Si}(p,x)\ce{^{22,24}Na} yields.
The observation of the \ce{^{22,24}Na} activities in Cu and Nb foils  represents an indirect measurement of the \ce{^{nat}Si}(p,x)\ce{^{22,24}Na} cross sections, but  will not be reported due to 
% the number of assumptions involved in such a calculation.
uncertainties in the areal density of the Si in the adhesive.
% The EoB \ce{^{22,24}Na} activities have been measured directly, but to convert these into absolute cross sections, accurate knowledge of the precise silicone composition and areal density are required.
% These have been taken as
However, if we assume a 10\% Si stoichiometric basis and an areal density of 4.79\,mg/cm$^2$ (based on bulk density),
% respectively, for the purposes of transport calculations, but this level of confidence is insufficient for the reporting of a cross section.
% In principle, it would be possible  to 
we can subtract out the measured \ce{^{22,24}Na} activity at each Nb and Cu foil position (correcting for the minor difference in proton energy between adjacent foils) from the apparent \ce{^{22,24}Na}  activities observed in each Al foil packet, in order to obtain the \enquote{true} or uncontaminated fluence via the Al monitor reactions, shown  
% The results of this  may be seen 
in \autoref{fig:na_subtraction}.
Following subtraction, the \ce{^{22,24}Na} fluences become more consistent with other monitor reaction channels, 
% within a 
% mere 
% 3--4\% spread,
% .
% While this would circumvent the assumptions needed for reporting \ce{^{nat}Si}(p,x)\ce{^{22,24}Na} cross sections, subtraction of  inaccurately quantified \ce{^{22,24}Na} activity in each Nb and Cu foil would propagate into the final fluence determination at each energy position, shifting the magnitude of all reported cross sections.
% Even following subtraction, 
though  \ce{^{22}Na} fluence remains 3--6\% higher than the weighted mean of the remaining monitor reaction channels.
While the dramatic improvement in monitor reaction consistency builds confidence, in the interest of surety and because they are consistent, only the \ce{^{nat}Cu}(p,x)\ce{^{56}Co}, \ce{^{nat}Cu}(p,x)\ce{^{62}Zn}, and \ce{^{nat}Cu}(p,x)\ce{^{65}Zn} monitor reaction channels will be used for fluence determination for the reported cross sections.
% In both cases, this disparity is caused by the fact that both of these monitor reactions may also form the \ce{^{22}Na} and \ce{^{56}Co} reaction products through contamination by secondary neutron (n,x) channels, increasing the apparent fluence as observed by these monitor reactions.
% Since no method for reliably separating the fraction of \ce{^{22}Na} and \ce{^{56}Co} activities induced through (n,x) exists, the fluences predicted by these monitor channels are not used in the final determination of the proton fluence seen by Nb foils. 
% The fact that this \enquote{extra fluence} diminishes at lower energy is likely attributed to the fact that the \ce{^{nat}Al}(p,x)\ce{^{22}Na} and \ce{^{nat}Cu}(p,x)\ce{^{56}Co} have energetic thresholds of 23.35 and 36.76 MeV, respectively. 
% The fraction of secondary neutrons produced by (p,xn)  which are energetic enough to populate the  \ce{^{22}Na} and \ce{^{56}Co} reaction products at the lower energy positions becomes progressively smaller.
This serves as a pointed example of the importance of selecting monitor reaction products inaccessible through channels aside from the primary reaction (\ce{^{nat}Al}(p,x)\ce{^{22,24}Na}, in this case ), as noted previously.
% However,  the fact that both monitor reactions measure consistently higher fluence than the other channels on each foil builds confidence that the monitor reactions accurately indicate the presence of a non-negligible secondary neutron flux.






% %%%
% %
% %  Moving this section to PhD thesis - too much detail on applications for an experimental paper
% %
% %%%
% 
% \ce{^{57}Ni} ($t_{1/2}=35.60\pm0.06$ h, $\epsilon$=100\% to \ce{^{57}Co} \cite{Bhat1998}), while useful on its own as a positron emitter, stands poised as a particularly promising candidate for theranostic pairing with the soft $\beta^-$ emitter \ce{^{66}Ni} ($t_{1/2}=54.6\pm0.3$ h, $\beta^-$=100\% to \ce{^{66}Cu} \cite{Browne2010a}) \cite{PMID:7632762,zweit1996medium,Graves2016,Rosch2014}. 
% \ce{^{nat}Cu}(p,x)\ce{^{57}Ni} would seem to be an intriguing production pathway, due to the ready availability of Cu metal as target, combined with the fact that production in this pathway strongly favors \ce{^{57}Ni} over \ce{^{56}Ni} --- indeed, the \ce{^{57}Ni}/\ce{^{56}Ni} 
% %ratio of cross sections is approximately 70 at 61.58 MeV, and varies from 11-18 at the 70-90 MeV positions.
% ratio of production rates is approximately 290 at 61.58 MeV, and varies from 45--75 at the 70--90 MeV positions.
% The traditional route for \ce{^{57}Ni} production is via \ce{^{nat}Co}(p,3n)\ce{^{57}Ni}, but at moderate energies, suffers from more  \ce{^{56}Ni} contamination than \ce{^{nat}Cu}(p,x)  ($\sigma_\text{57Ni} / \sigma_\text{56Ni}\approx$ 10 at maximum).
% Lower-energy production via \ce{^{nat}Co}(p,3n) at 24--40\,MeV is below threshold for \ce{^{56}Ni}, but has a peak cross section of approximately 10 mb, making \ce{^{nat}Cu}(p,x) the superior production route for moderate-energy accelerators  \cite{MICHEL1997153,Ditrói2013}.
% 
% 
% % Moving away from the potential PET emitters, 
% \ce{^{64}Cu}  ($t_{1/2}$ = 12.7 h) undergoes $\beta^+$ decay (61.5\% branching ratio) to \ce{^{64}Ni} or $\beta^-$ decay (38.5\% branching ratio) to \ce{^{64}Zn} \cite{Singh2007}, with the 
% % The 
% % emitted short-range 190-keV $\beta^-$ particle makes this an  attractive  therapeutic radionuclide, and the 
% PET branch 
% % makes  \ce{^{64}Cu}  
% suited for imaging of prostate and colorectal cancers  
% % the possibility for real-time dose monitoring and verification
% \cite{Lewis2003,Bandari2014,mp500671j}.
% % This makes \ce{^{64}Cu} particularly desirable  for emerging radiation therapy protocols \cite{Lewis2003,Bandari2014,mp500671j}.
% Several production routes currently exist: \ce{^{64}Ni}(p,n)  uses 8--14 MeV protons on the expensive enriched target \ce{^{64}Ni} (0.9255\% natural abundance), but offers a high radioisotopic purity assuming a highly enriched target \cite{Szelecsenyi1993,Aslam2009}.
% \ce{^{68}Zn}(p,$\alpha$n) requires more energetic 20--30 MeV protons, and necessitates an enriched target (18.45\% natural abundance) to avoid the co-production of radio-copper impurities \cite{Hilgers2003,Szelecsenyi2005}.
% More recently, the use of compact DD neutron generators for \ce{^{nat}Zn}(n,p) production has been proposed, with the promise of mCi-scale production with high specific activity  \cite{Voyles2017}.
% % \ce{^{nat}Cu}(p,x)\ce{^{64}Cu} could be another potential production pathway 
% 
% 
% 
% \ce{^{86}Y} ($t_{1/2}=14.74\pm0.02$ h, $\epsilon$=100\% to \ce{^{86}Sr}  \cite{NEGRET20151}) is another novel  emerging  PET isotope, whose longer half-life has poised it for applications as a tracer for slower metabolic processes, as well as in   pharmacokinetics studies \cite{Valdovinos2017,Nickles2003,Qaim2008,QaimSyedM2011}.
% In particular, it is highly desired to form a theranostic pair with the widely-employed $\beta^-$ therapy agent \ce{^{90}Y} ($t_{1/2}=64.00\pm0.21$ h, $\beta^-$=100\% to \ce{^{90}Zr} \cite{Browne1997}), which can be produced from a long-lived \ce{^{90}Sr} generator and emits no discrete observable gamma-rays or x-rays though decay \cite{Herzog1993}.
% Although a weak positron branch exists and bremsstrahlung scintigraphy is commonly used for clinical imaging of the \ce{^{90}Y} biodistribution, theranostics  necessitate an imaging isotope to be paired for quantification of its uptake and biodistribution  \cite{Nickles2004}.
% Conventional production of \ce{^{86}Y} proceeds through low-energy (7--14 MeV) irradiation via \ce{^{86}Sr}(p,n), which requires an enriched  \ce{^{86}Sr} target (9.86\% natural abundance), in order to eliminate contamination from (p,n) on the other stable \ce{^{84,87,88}Sr} isotopes  \cite{Rosch1993}.
% Alternatively, production at 33--43 MeV via \ce{^{88}Sr}(p,3n) has been proposed --- this pathway also requires an enriched target (82.58\% natural abundance) for the same reason, but contamination with other Y co-activities will be even more pronounced than via (p,n), due to the opening of (p,n) and (p,2n) channels on all stable Sr isotopes \cite{doi:10.1139/v67-193,levkovski1991cross}.
% Minimizing activity from other isotopes of the element in question is essential for producing radionuclides in high specific activity, as these competing isotopes are often impractical to separate out by radiochemical means.
% % \comment{Stephen:  impossible by radiochemical means, and impossible by affordable/practical means.\\Substitute for ``difficult'' post-discussion.}
% As a result, it would appear that \ce{^{nat}Nb}(p,x) is a poor route for  \ce{^{86}Y} production in this respect, as it only reaches a maximum of approximately 35\% radioisotopic purity.
% The  dominant yttrium radioisotope produced by  \ce{^{nat}Nb}(p,x) in the 40--90 MeV region is  \ce{^{87}Y} ($t_{1/2}=79.8\pm0.3$ h, $\epsilon^-$=100\% to \ce{^{87m}Sr} \cite{Johnson2015}).
% However,  \ce{^{87}Y} itself has application as a generator for  \ce{^{87m}Sr} ($t_{1/2}=2.815\pm0.012$ h, IT=99.70\% to \ce{^{87}Sr} \cite{Johnson2015}), which is used for imaging studies of metastatic bone cancers, especially when in a theranostic pair with the established therapy agent \ce{^{89}Sr} ($t_{1/2}=50.563\pm0.0025$ d, $\beta^-$=100\% to \ce{^{89}Y} \cite{Singh2013}) \cite{Kiselev1974,Kandil2009}.
% % Since the radio-yttrium purity of \ce{^{87}Y} is approximately 88\% between 51--61 MeV in \ce{^{nat}Nb}(p,x), this could present an intriguing route for  \ce{^{87}Y} production.
% 
% 
% 
% 
% \ce{^{89}Zr} ($t_{1/2}=78.41\pm0.12$ h, $\epsilon$=100\% to \ce{^{89}Y}  \cite{Singh2013}) is a long-lived positron emitter useful as a tracer for slow biological processes, immune studies, and imaging of liver and  breast cancers \cite{Verel2003,Dijkers2009,Dijkers2010}.
% Current production focuses on \ce{^{89}Y}(p,n)\ce{^{89}Zr} between 9--14 MeV, which offers an extremely high-purity route on a mono-isotopic target and a strong population of \ce{^{89}Zr}, with a peak cross section of nearly 800 mb   \cite{PhysRevC.38.1624,Omara2009}.
% Due to co-production of additional \ce{^{86,87,88}Zr} radio-zirconium,  \ce{^{nat}Nb}(p,x) is clearly inferior to  \ce{^{89}Y}(p,n)\ce{^{89}Zr}, as the Nb route has a smaller peak cross section of approximately 290 mb, and achieves only 10--20\% radioisotopic purity in the 50--90 MeV region.
% 
% 
% 
% 
% \ce{^{90}Nb} ($t_{1/2}=14.60 \pm 0.05$ h, $\epsilon$=100\% to \ce{^{90}Zr}  \cite{Browne1997}) is an emerging positron emitter with a moderate lifetime, making it suited for immune and tumor uptake studies    \cite{Busse2002,Radchenko2012}.
% It is typically produced using 8--15 MeV protons via \ce{^{90}Zr}(p,n)\ce{^{90}Nb}, using an enriched target (51.45\% natural abundance) for high radioisotopic purity, and produces a product with minimal contamination and a peak cross section of approximately 750 mb  \cite{Busse2002}.
% \ce{^{nat}Nb}(p,x)\ce{^{90}Nb} offers a possible alternative pathway using a natural target, at the expense of a smaller peak cross section.
% \ce{^{90}Nb} may be produced directly with an approximately 370 mb peak cross section and 99\% radioisotopic purity, or could be produced as a \ce{^{90}Mo}/\ce{^{90}Nb} generator, which would have nearly 100\% radioisotopic purity by using protons below the \ce{^{nat}Nb}(p,5n) threshold of 45.76 MeV.
% However, the greatest problem with using the \ce{^{nat}Nb}(p,x) reaction to produce \ce{^{90}Nb} is the inability to separate the radioisotope from the target itself, rendering the production of a high-specific activity product impossible.  
% 
% 
% 
% 
% 
% 
% 
% Finally, \ce{^{82\text{m}}Rb} ($t_{1/2}=6.472\pm0.006$ h, $\epsilon$=100\% to \ce{^{82}Kr}  \cite{Tuli2003}) is a diagnostic and emerging Auger-therapy agent, typically seen as a contaminant in \ce{^{82}Sr}/\ce{^{82}Rb} generators 
% % for PET studies
% \cite{Kovacs1991}.
% It is commonly produced via \ce{^{82}Kr}(p,n) at 10--15 MeV, using an enriched \ce{^{82}Kr} gaseous target, with a peak cross section of approximately 400 mb at 12 MeV \cite{Kovacs1991}.
% Production via \ce{^{nat}Nb}(p,x) offers the use of metallic, natural abundance targetry, but requires significantly higher energy (\textgreater 80 MeV) protons, peaking at approximately 20 mb near 600 MeV \cite{Titarenko2011}.
% It is clear that this production route offers no advantage over existing \ce{^{82}Kr}(p,n) routes for in-house production.


% Example text from template file
% 
% \subsection{Promenade Exeter}
% 
% Inertia breakup Brookline.  Hebrew, prexy, and Balfour.  Salaam
% applaud, puff teakettle.
% 
% \begin{quote}
% Ugh servant Eulerian knowledge Prexy Lyman zig wiggly.  Promenade
% adduce.  Yugoslavia piccolo Exeter.  Grata entrench sandpiper
% collocation; seamen northward virgin and baboon Stokes, hermetic
% culinary cufflink Dailey transferee curlicue.  Camille, Whittaker
% harness shatter.  Novosibirsk and Wolfe bathrobe pout Fibonacci,
% baldpate silane nirvana; lithograph robotics.  Krakow, downpour
% effeminate Volstead?
% \end{quote}
% 
% Davidson witting and grammatic.  Hoofmark and Avogadro ionosphere.
% Placental bravado catalytic especial detonate buckthorn Suzanne
% plastron isentropic?  Glory characteristic.  Denature?  Pigeonhole
% sportsman grin historic stockpile.  Doctrinaire marginalia and art.
% Sony tomography.  Aviv censor seventh, conjugal.  Faceplate emittance
% borough airline.  Salutary.  Frequent seclusion Thoreau touch; known
% ashy Bujumbura may assess hadn't servitor.  Wash, Doff, and Algorithm.
% 
% \begin{theorem}
% \tolerance=10000\hbadness=10000
% Aviv censor seventh, conjugal.  Faceplate emittance borough airline.  
% Salutary.
% \end{theorem}
% 
% Davidson witting and grammatic.  Hoofmark and Avogadro ionosphere.
% Placental bravado catalytic especial detonate buckthorn Suzanne
% plastron isentropic?  Glory characteristic.  Denature?  Pigeonhole
% sportsman grin historic stockpile. Doctrinaire marginalia and art.
% Sony tomography.  Aviv censor seventh, conjugal.  Faceplate emittance
% borough airline.  Salutary.  Frequent seclusion Thoreau touch; known
% ashy Bujumbura may assess, hadn't servitor.  Wash, Doff, Algorithm.
% 
% \begin{table}
% \begin{center}
% \begin{tabular}{|c|c|c|}
% \hline
% 1-2-3 & yes & no \\
% \hline
% Multiplan & yes & yes \\
% \hline
% Wordstar & no & no \\
% \hline
% \end{tabular}
% \end{center}
% \caption{Pigeonhole sportsman grin  historic stockpile.}
% \end{table}
% Davidson witting and grammatic.  Hoofmark and Avogadro ionosphere.
% Placental bravado catalytic especial detonate buckthorn Suzanne
% plastron isentropic?  Glory characteristic.  Denature?  Pigeonhole
% sportsman grin historic stockpile. Doctrinaire marginalia and art.
% Sony tomography.
% 
% \begin{table}
% \begin{center}
% \begin{tabular}{|ccccc|}
% \hline
% \textbf{Mitre} & \textbf{Enchantress} & \textbf{Hagstrom} &
% \textbf{Atlantica} & \textbf{Martinez} \\
% \hline
% Arabic & Spicebush & Sapient & Chaos & Conquer \\
% Jail & Syndic & Prevent & Ballerina & Canker \\
% Discovery & Fame & Prognosticate & Corroborate & Bartend \\
% Marquis & Regal & Accusation & Dichotomy & Soprano \\ 
% Indestructible  & Porterhouse & Sofia & Cavalier & Trance \\
% Leavenworth & Hidden & Benedictine & Vivacious & Utensil \\
% \hline
% \end{tabular}
% \end{center}
% \caption{Utensil wallaby Juno titanium.}
% \end{table}
% 
% Aviv censor seventh, conjugal.  Faceplate emittance borough airline.
% Salutary.  Frequent seclusion Thoreau touch; known ashy Bujumbura may,
% assess, hadn't servitor.  Wash\cite{cmusic}, Doff, and Algorithm.
% 
% \begin{figure}
% \[ \begin{picture}(90,50)
%   \put(0,0){\circle*{5}}
%   \put(0,0){\vector(1,1){31.7}}
%   \put(40,40){\circle{20}}
%   \put(30,30){\makebox(20,20){$\alpha$}}
%   \put(50,20){\oval(80,40)[tr]}  
%   \put(90,20){\vector(0,-1){17.5}}
%   \put(90,0){\circle*{5}}
% \end{picture}
%  \]
% \caption{Davidson witting and grammatic.  Hoofmark and Avogadro ionosphere.  
% Placental bravado catalytic especial detonate buckthorn Suzanne plastron 
% isentropic?  Glory characteristic.  Denature?  Pigeonhole sportsman grin.}
% \end{figure}
% 
% Davidson witting and grammatic.  Hoofmark and Avogadro ionosphere.
% Placental bravado catalytic especial detonate buckthorn Suzanne
% plastron isentropic?  Glory characteristic.  Denature?  Pigeonhole
% sportsman grin historic stockpile. Doctrinaire marginalia and art.
% Sony tomography.  Aviv censor seventh, conjugal.  Faceplate emittance
% borough airline.\cite{fm} Salutary.  Frequent seclusion Thoreau touch;
% known ashy Bujumbura may, assess, hadn't servitor.  Wash, Doff, and
% Algorithm.
% 
% \begin{itemize}
% \item Davidson witting and grammatic.  Jukes foundry mesh sting speak,
% Gillespie, Birmingham Bentley.  Hedgehog, swollen McGuire; gnat.
% Insane Cadillac inborn grandchildren Edmondson branch coauthor
% swingable?  Lap Kenney Gainesville infiltrate.  Leap and dump?
% Spoilage bluegrass.  Diesel aboard Donaldson affectionate cod?
% Vermiculite pemmican labour Greenberg derriere Hindu.  Stickle ferrule
% savage jugging spidery and animism.
% \item Hoofmark and Avogadro ionosphere.  
% \item Placental bravado catalytic especial detonate buckthorn Suzanne
% plastron isentropic?
% \item Glory characteristic.  Denature?  Pigeonhole sportsman grin
% historic stockpile.
% \item Doctrinaire marginalia and art.  Sony tomography.  
% \item Aviv censor seventh, conjugal.
% \item Faceplate emittance borough airline.  
% \item Salutary.  Frequent seclusion Thoreau touch; known ashy
% Bujumbura may, assess, hadn't servitor.  Wash, Doff, and Algorithm.
% \end{itemize}
% 
% Davidson witting and grammatic.  Hoofmark and Avogadro ionosphere.
% Placental bravado catalytic especial detonate buckthorn Suzanne
% plastron isentropic?  Glory characteristic.  Denature?  Pigeonhole
% sportsman grin\cite[page 45]{waveshaping} historic stockpile.
% Doctrinaire marginalia and art. Sony tomography.  Aviv censor seventh,
% conjugal. Faceplate emittance borough airline.  Salutary.  Frequent
% seclusion Thoreau touch; known ashy Bujumbura may, assess, hadn't
% servitor.  Wash, Doff, and Algorithm.
% 
% \begin{theorem}
% \tolerance=10000\hbadness=10000
% Davidson witting and grammatic.  Hoofmark and Avogadro ionosphere.  
% Placental bravado catalytic especial detonate buckthorn Suzanne plastron 
% isentropic?
% \end{theorem}
